\section{Control Units}\label{s:cont}
Before it is possible to fly the drone up in a certain altitude and move it around in the four directions, a method of controlling the drone is necessary. The drone can either be controlled with a remote control unit or by on-drone software. %hard coded to follow a path. 
\newline
There are different options when it comes to controlling the drone with a control unit. It is firstly important to understand how communication between a drone and a control unit works. %drone controlling works.

The communications works by the control unit having a built-in transmitter. By using a control unit the transmitter can transfer control data to the drone. The data could be anything that the user presses on the control unit. When the data are sent, the built-in receiver on the drone, will receive all the data from the control unit. To use this method it is important the control unit and the drone are connected to each other \cite{Control}.
%The way the user communicates with the drone is by sending data to the drone with a transmitter like a remote control. The data will be received by the receiver equipped on the drone, and the drone will respond to it. \cite{control}



The control units there are most used for drones are handheld remote controllers, using either Bluetooth, Wi-Fi or a proprietary RF protocol.

\subsubsection*{Proprietary RF}
%As the name also says 
The communication between the transmitter and receiver is carried out over a proprietary RF protocol, commonly in the license free 2.4GHz frequency band. Transceivers of this type often incorporate frequency hopping and error correction, and high powered transmitters, to ensure a stable, long range connection can be maintained. 

\subsubsection*{Bluetooth/Wi-Fi controllers}
The drone can also be controlled via an application using Bluetooth or Wi-Fi compatible devices like smartphones and tablets. Manufactures of drones like DJI have developed an application that gives advanced positioning, first person video controls, programmable flight routes, and much more. The downsides of using Bluetooth or Wi-Fi as a communication method, are the lack of a standard protocol for control drones and the short range \cite{droneRange}. 

Wi-Fi provides the ability to transmit larger amounts of data to and from the drone within a specific radius. The problem with using Wi-Fi is the short range of communication, in many cases lose connection. 

\subsection*{Autonomous drone control}
A way for the drone to fly automatically is by using Global Position system (GPS), it gives the ability such as auto pilot features, since it provides the drone with accurate position data.
Since the GPS is limited for outdoor applications, it is not useable indoors, and there are not many companies that have attempted to develop alternative systems for indoor purposes. An example of an indoor positioning system is GT-position from Gamesontrack \cite{gt-position}. 

GT-position indoor positioning-system uses Gameontracks own indoor satellites.
Based on a combination of radio and ultrasound they provide precise distances to any moving or stationary unit which has a sender in the system.
The satellites provide precision up to 10 mm and can measure up to 8 m distance. Satellites are combined in scenarios of 2, 3 or more which together form a 2D or 3D position in a coordinate system. More scenarios can be combined to extend coverage \cite{gt-position}.

\subsection*{Conclusion} 
For the purpose of this project RF controller will be used, because it exposes a standardized interface which can be broken into. The drone chosen for this project is equipped with an RF receiver, which can communicate with the RC transmitter in the controller \cite{Control}. Also mentioned earlier, by controlling the drone with RF, it provides control of the drone for a much greater distance than a Wi-Fi or Bluetooth connection.











