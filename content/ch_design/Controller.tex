 \begin{figure}[H]
     \centering
     \includegraphics[width=\textwidth]{figures/ch_design/controller/ControlDiagramTF.pdf}
     \caption{Block diagram illustration of the control system  to control the levitation of the Quadcopter.}
     \label{fig:controlDiagram}
 \end{figure}

%To avoid steady state error, it is necessary to examine PID controller, which stands for proportional integrate derivative controller.
To make the full control loop, there are needed two more components apart from the transfer function. This can be seen in figure \ref{fig:controlDiagram}. From this figure it is visual that there is needed a function for the sensors and a controller of the control loop. 
First the function for the sensor will be found and afterward the PID controller will be examined with all the components.


\section{Feedback control loop}\label{s:feedback_loop}
To design a control system, a feedback control loop is needed. The sensor used in on the drone has a function of 1:1 sensing, but since it is a discrete time sampled sensor, a phase-shift is introduced, described in equation \ref{eq:formular_sampling_feedback}. \ref{eq:formular_sampling_feedback} \cite{digital_control}.
\begin{equation}\label{eq:formular_sampling_feedback}
    H(s)=\frac{2/T}{s+2/T}
\end{equation}
In this equation the T is the sampling time for the sensor, in this case the sampling time is $25\ ms$, so the expression for the feedback loop will be as seen in equation \ref{eq:feedback_loop}\cite{feedback_control}.
\begin{equation}\label{eq:feedback_loop}
    H(s)=\frac{\frac{2}{0.0025}}{s+\frac{2}{0.0025}}
\end{equation}

\section{PID controller}
To figure out what kind of controller there are needed in the system, to get the right output of the system. In this case the PID controller of a first order close loop system will be explained to find the best suited controller for the system. In the PID controller, there are both a proportional part, an integral part and a derivative part. All these three parts have influence on the output signal. 

\begin{figure}[H]
    \centering
    \includegraphics[width=0.9\textwidth]{figures/ch_design/PIDController/PIDControl.pdf}
    \caption{Illustration of a PID controller connected in a control loop.}
    \label{fig:PID_Controller}
\end{figure}
As shown in figure \ref{fig:PID_Controller} the focus for this control system, will be on a closed loop control system. This is used because of the implementation of the sensors in the system. To control this system a PID controller will be used, or a part of it. First the different parts of the PID controller will be explained. The three parts that will be explained are the proportional part, the integral part and the derivative part \cite{digital_control}.

\subsection*{Proportional controller}
The proportional control is in most cases the main driving force in a controller. The proportional control changes its output in proportional to the error. 
The function for the proportional correction, is shown in equation \ref{eq:kp}.
\begin{equation}\label{eq:kp}
    \frac{U(s)}{E(s)}=K_p
\end{equation}
In equation \ref{eq:kp}, the E(s) function is the error between the Reference, and the measured output U(s) as seen on figure \ref{fig:controlDiagram}.
If the $K_p$ gets increased, it will cause the proportional part of the regulator to have a larger influence on the system being regulated.
If a too big $K_p$ value is chosen it can cause the system being regulated to oscillate and eventually become unstable, as a larger $K_p$ will reduce the phase margin for the system. If the $K_p$ is too low, the control loop will not respond enough to disturbances or set point changes. The proportional controller is used for set the phase margin and gain margin to the optimal setting \cite{digital_control}. 


\begin{comment}


The following graph \ref{fig:P_Control} will illustrate a proportional control, and what will happen, when a drone rises to a giving altitude, which in this case it will be 40 cm. 
\begin{figure}[H]
    \centering
    \includegraphics[width=0.8\textwidth]{figures/ch_design/PIDController/ProportionalGraph.png}
    \caption{Illustration of a proportion control system.}
    \label{fig:P_Control}
    \end{figure}
The graph \ref{fig:P_Control} at the starting point there is an error of 40 cm, since there is large of error, it will generate a large propeller speed, which the drone will rise. In step 1 at 40 cm, the error is equal 0, and at this point the propeller speed decrease and the drone will fall back to earth. However, since the propeller speed is large, it will have overshoot as seen in the graph (step 1). When it is under 40 cm the propeller speed will increase again and since the error is 1 and is not large, it will increase the propeller speed slightly as seen at graph (step 2). To hover the drone, its need a certain propeller speed, where the lifting force is exactly equal to the weight of the drone, that speed will hover the drone.    

To hover the drone at certain altitude will depend on the controller gain and propellers speed. In this example we assume that at 80 rpm the drone will hover, if the proportional gain is 2 and the drone is at starting point, which the error is 40 cm, the drone would hover right at the ground level since 2 \(\times\) 40 is 80 rpm. The following graph \ref{gra:P_Control_graph} illustrates where the drone will hover, when the gain increases.

\begin{figure}[H]
    \centering
    \includegraphics[width=0.6\textwidth]{figures/ch_design/PIDController/ProportionalGraph1.png}
    \caption{Proportional control graph.}
    \label{gra:P_Control_graph}
    \end{figure}
The graph \ref{gra:P_Control_graph} at step one the proportional gain is set to 4, which the error is reduced to 20 cm, since the drone is closer to the set point. In step two the proportional gain is 80 and the drone is hovering at 39 cm, which the error is 1 cm. This constant error is also called steady state error and this will never be eliminated in proportional control. There is a need of a or control mode, which can use past information to eliminate the steady state error. 

\end{comment}

\subsection*{Integral controller}
Apart from the proportional controller, another controller is an integral controller. The integral controller works on the sum of all past errors, to eliminate steady-state errors. This means that the integral controller has infinite DC-gain on the error. The ways this controller works is by continuously summing the past errors and multiplying this sum with the $k_I$ gain.

The error correcting time can depend on the size of the error, as a larger error will add or subtract more from the sum. The time used is defined be the size of $k_I$ in the equation, can also be seen in equation \ref{eq:integral_control} \cite{digital_control}. 

\begin{equation}\label{eq:integral_control}
    \frac{U(s)}{E(s)}=\frac{k_I}{s} \to u(t)=k_I \int_{t_0}^t e(\tau) \ d\tau
\end{equation}
In this equation $k_I$ is the integral constant. It can be seen from the equation, that the size of $k_I$ has influence on the reaction time. If this value is small the reaction time will be fast, but if it is a big value it will have the opposite effection. If the time is set to small the control part can be slow, but if the time is set to big the control loop can begin to oscillate and become unstable. The upside of using a integral controller, is to add more gain for the transfer function. By doing this the steady-state behavior will be improved along side the adding of gain. 
If the $k_I$ get to big it can also increase the overshoot for the control loop\cite{digital_control}. 


\subsection*{Derivative controller}
The way the derivative controller works is by finding the rate of change of the error. By doing this the derivative control, can be said to work on future errors, and correcting them by look at the past errors to get the rate they have changed. This means the controller corrects errors at a high rate, and if there are no errors the derivative controller will be zero. The equation for the derivation controller can be seen in equation \ref{eq:derivative_control} \cite{digital_control}.

\begin{equation}\label{eq:derivative_control}
\frac{U(s)}{E(s)}=k_D \cdot s
\end{equation}

In this equation the $k_D$ is the derivative gain. The larger the derivative gain is, the higher the influence of the controller will be, but if the gain is set too high the controller will begin to oscillate and become unstable. And if the derivative gain is to small the influence on the output will be less \cite{digital_control}. By implementing a derivative controller, an increase in phase margin can be obtained. By increasing the phase margin a damping of the transfer function is achieved.

A disadvantage of the derivative controller is that, as it works on rate of change, and higher rate of change gives a higher response, it amplifies high frequencies more. This happens to be where most natural noise is found, and as such, it amplifies noise in a system.

\section{Differences between the PID-controllers parts}\label{s:different_pid}
To compare the different combination between the PID-controller's parts, first a bode plot of all the components will be made to compare how they infect the bode plot. This bode plot can be seen figure \ref{fig:PID_bode}.

\begin{figure}[H]
    \centering
    \includegraphics[width=\textwidth]{figures/ch_design/controller/Bodeplot_PID.pdf}
    \caption{An example of the different bode plot for the different PID-control part, to illustrate how they work on the gain and phase, both individual and combine with another.}
    \label{fig:PID_bode}
\end{figure}
From this figure it can be seen how the gain and phase is affected by the changes in the $K_p$, $k_I$ and $k_D$. It is also visible to see that the integral controller affects the gain but not the phase, only when it is combined with the proportional controller. 
It can also be seen that the derivative controller change the phase but not the gain as much. All these combinations can be used in different ways, but first for this project a proportional controller will be calculated, this will be done in the next section.

% http://blog.opticontrols.com/archives/344
\section{Design of proportional controller}\label{sec:design_controller}
As it has been decided to use a proportional controller to regulate the system, the proportional gain $K_p$ has to be found. 
When finding the value of $K_p$ the requirements from chapter \ref{ch:Req} have to be taken into consideration. 
The requirements that have to be taken into consideration are:
\begin{itemize}
    \item Overshoot of max 30\%
    \item Settling time of  max 10 seconds
    \item Steady state of $\pm$5\%
\end{itemize}
To find $K_p$ we have to start finding the damping factor. From the allowed overshoot the damping ratio $\zeta$ can be found with the equation \ref{eq:dampingRatio} \cite{nise2007control}.
\begin{equation}\label{eq:dampingRatio}
    \zeta = \frac{-\ln(overshoot/100)}{\sqrt{\pi^2+\ln^2(overshoot/100)}}
\end{equation}
With a overshoot of 30\% the damping ratio are found to be 0.358, and from this ratio the phase margin $\Phi_M$ can then be found with the use of the equation \ref{eq:phaseMargin} \cite{nise2007control}.
\begin{equation}\label{eq:phaseMargin}
    \Phi_M = \arctan\left(\frac{2\zeta}{\sqrt{-2\zeta^2+\sqrt{1+4\zeta^4}}}\right)
\end{equation}
The phase margin is calculated to 0.682 radian which are equal to $39.09\degree$. By making a bode plot of the open loop term ($G(s)H(s)$) from  the diagram in figure \ref{fig:controlDiagram} can the gain adjustment be fund.
In figure \ref{fig:des_bodeplot} can the bode blot of $G(s)H(s)$ be seen, where the gain margin, at the phase margin of $39.09\degree$, can be fund to be 39.7 dB.
\begin{figure}[h]
    \centering
    \includegraphics[width=0.8\textwidth]{figures/ch_design/controller/bodeplot.png}
    \caption{Bode plot of the open loop term ($G(s)H(s)$) with the phase margin of 39.09$\degree$ and the corresponding gain margin.} 
    \label{fig:des_bodeplot}
\end{figure}
\newline
To get the value of $K_p$ the magnitude have to be adjusted so the gain margin are 0 dB, this can be done by decreasing the calculated gain margin of 39.7 dB.
The -39.7 dB have to be converted to a gain factor, which have been done by using the equation \ref{eq:db2kp} and calculated the factor to 0.01.
\begin{equation}\label{eq:db2kp}
    10^\frac{dB}{20}
\end{equation}
This mean $K_p$ is 0.01.
With $K_p$ found the control diagram of the system get to be as seen in figure \ref{fig:dec_Final_block_diagram}.

\begin{figure}[H]
    \centering
    \includegraphics[width=\textwidth]{figures/ch_design/controller/FinalControlDiagram.pdf}
    \caption{Block diagram of the final control system.}
    \label{fig:dec_Final_block_diagram}
\end{figure}






